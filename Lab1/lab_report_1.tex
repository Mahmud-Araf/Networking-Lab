\documentclass[11pt]{article}

\usepackage{graphicx}
\usepackage{url}

\begin{document}

\begin{titlepage}
    \begin{center}
        \includegraphics[scale=0.50]{du.jpeg}\par
        \begin{Huge}
            \textsc{University of Dhaka}\par
        \end{Huge}
        \begin{Large}
            Department of Computer Science and Engineering\par \vspace{1cm}
            CSE-3111 : Computer Networking Lab \\[12pt]
            Lab Report 1 : Lab exercises on LAN configuration and troubleshooting tools
        \end{Large}
    \end{center}
    \begin{large}
        \textbf{Submitted By:\\[12pt]}
        Name : Zisan Mahmud\\[8pt]
        Roll No : 23\\[12pt]
        Name :  Abdullah Al Mahmud\\[8pt]
        Roll No : 15\\[12pt]
        \textbf{Submitted On : \\[12pt]}
        February 2, 2023\\[20pt]
        \textbf{Submitted To :\\[12pt]}
        Dr. Md. Abdur Razzaque\\[12pt]
        Dr. Md Mamunur Rashid\\[12pt]
        Dr. Muhammad Ibrahim\\[12pt]
        Mr. Md. Redwan Ahmed Rizvee
    \end{large}
\end{titlepage}

\section{Introduction}

The primary objective of this lab is to familiarize with few network troubleshooting tools like PING, TRACEROUTE, IFCONFIG, ARP, RARP, NSLOOKUP, NETSTAT etc. These tools are a necessity for every network administrator.

\section{Objectives}
\begin{itemize}
    \item
    \item
    \item
\end{itemize}
%%%
%%%
\section{Theory}



\section{Methodology}

\subsection{PING}
\subsection{TRACEROUTE}
\subsection{IFCONFIG}
\subsection{ARP}
\subsection{RARP}
\subsection{NSLOOKUP}
\subsection{NETSTAT}

\section{Experimental Result}

\section{Experience}

% \begin{thebibliography}
%     \bibitem{}
%     \bibitem{}
%     \bibitem{}
% \end{thebibliography}

\end{document}